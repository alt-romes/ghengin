\documentclass{article}

\title{Ghengin: A type-heavy, shader-centric Haskell game engine}
\author{Rodrigo Mesquita}

\begin{document}
\maketitle

\section{Introduction}

\begin{itemize}
  \item A render packet definition that validates the mesh and material of the
    render packet across the render pipeline it will be rendered in.
  \item Pattern matching on existential materials through type reps
\end{itemize}

\section{Render Pipeline}

\section{Meshes}

\section{Materials}

A \textbf{Material} is usually regarded as a core component of a rendering
engine which describes surface properties that define the visual appearence of
meshes when rendered. Material properties can include the surface color,
texture, parameters of lighting such as specularity, or custom parameters for a
custom shader.

We define a Material as a collection of properties that each \emph{render
packet} has. This collection of properties is passed to the shader programs
every frame and will ultimately define how each render packet is rendered. In
contrast to existing material systems, ...

\section{Render Packets}

% TODO: Render keys for pipelines are TypeReps ;)

\end{document}

